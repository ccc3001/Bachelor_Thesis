\documentclass[acmsmall, review]{acmart}
\usepackage{graphicx} % Required for inserting images
\usepackage[utf8]{inputenc}

\title{Deep learning based grading of motionartifacts in HR-pQCT}
\setcopyright{none}
\date{October 2022}
\author{Johann Strunck}
\email{johann.strunck@tuhh.de}
\begin{document}
	%to get a sinogram into image form it is necessary to use the inversed radon transform 
	%filtered back projection is an intresting algorithm for generating the ct images 
	%Cohens Kappa ist ein statistisches Maß für die Interrater-Reliabilität
	%F1 Score
	%seems like other algorithms have a hard time to distinguish motion artifacts that are based arund the middle(3)
	%Gradient-weighted CAM (Grad-CAM) have been proposed to makethe CNN models more transparen
	%Saliency maps visualise attention by computing the gradient of the output category with respect to input image
	%global average pooling which acts as a structural regularizer, preventing overfitting during training.
	%visuallizing CNNs global average pooling(GAP)
	%GAP loss encourages the network to identify the extent of the object as compared to GMP which encourages it to identify just one discriminative part
	% (CAM) Class activation maps allow us to visualizethe predicted class scores on any given image, highlightingthe discriminative object parts detected by the CNN
	%mini batch gradient descent 
	%Momentum:  this technique is employed in the objective function. It enhances both the accuracy and the training speed 
	%Adaptive Moment Estimation (Adam) -> latest trend in deep learning optimization
	%Moreover, the majority of the crowdsourcing workers are unable to make accurate notes on medical or biological images due to their lack of medical or biological knowledge. Thus, ML researchers often rely on field experts to label such images; however, this process is costly and time consuming. Therefore, producing the large volume of labels required to develop flourishing deep networks turns out to be unfeasible.
	%CBMA might be a good addition to my algorithm seems to imrpove overall performance
\begin{abstract}

\end{abstract}
	\maketitle

\section{Introduction}
	
	
\section{Literature review}
\subsection{Convolutional Neural Networks}
\subsection{Meidcal Imaging }
\subsection{statistical approach}
\subsection{Machine learning approach}


\section{Methodology}
\subsection{Methodology}
\subsection{Imroved Adaptive Moment Estimation (Adam)}
\subsection{Gaussian Noice}
\begin{comment}
	adding annealed Gaussian noise to the gradien
	-	surprisingly effective
	-	experiments indicate that adding annealed Gaussian noise by decaying the variance works better than using fixed Gaussian noise
	-	-adding n
	-	-random restarts and the use of a momentum-based optimizer like Adam are not sufficient to achieve the best results in the absence of added gradient noiseoise to the gradient helps in achieving higher average and best accuracy	content...
\end{comment}

\subsection{Batch Normalization}
\subsection{Data Augmentation}


\subsection{Dropout}

\subsection{ELU / ReLU}
\subsection{Maxout Unit}
\subsection{CAM / Grad-CAM}
\subsection{Transfer Learning}
\subsection{Bayesian Approaches}
\subsection{Network In Network (NIN)}
\subsection{Convolution Block Attention Model(CBAM)}
\subsection{Very Deep Constitutional Networks}
\subsection{Data Distribution}
\subsection{CNN Structure}


\section{Results}
\subsection{F1 Score}
\subsection{Grad-CAM}
\subsection{False Positive Rate}
\subsection{Accuracy}
\subsection{Sensitivity}
\section{Discussion/Conclusion}
\end{document}
