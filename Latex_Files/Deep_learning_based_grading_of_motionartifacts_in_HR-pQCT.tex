\documentclass[acmsmall, review]{acmart}
\usepackage{graphicx} % Required for inserting images
\usepackage[utf8]{inputenc}
\title{Deep learning based grading of motionartifacts in HR-pQCT}
\setcopyright{none}
\date{October 2022}
\author{Johann Strunck}
\email{johann.strunck@tuhh.de}
\begin{document}
	%to get a sinogram into image form it is necessary to use the inversed radon transform 
	%filtered back projection is an intresting algorithm for generating the ct images 
	%Cohens Kappa ist ein statistisches Maß für die Interrater-Reliabilität
	%F1 Score
	%seems like other algorithms have a hard time to distinguish motion artifacts that are based arund the middle(3)
	%Gradient-weighted CAM (Grad-CAM) have been proposed to makethe CNN models more transparen
	%Saliency maps visualise attention by computing the gradient of the output category with respect to input image
	%global average pooling which acts as a structural regularizer, preventing overfitting during training.
	%visuallizing CNNs global average pooling(GAP)
	%GAP loss encourages the network to identify the extent of the object as compared to GMP which encourages it to identify just one discriminative part
	% (CAM) Class activation maps allow us to visualizethe predicted class scores on any given image, highlightingthe discriminative object parts detected by the CNN
	%mini batch gradient descent 
	%Momentum:  this technique is employed in the objective function. It enhances both the accuracy and the training speed 
	%Adaptive Moment Estimation (Adam) -> latest trend in deep learning optimization
	%Moreover, the majority of the crowdsourcing workers are unable to make accurate notes on medical or biological images due to their lack of medical or biological knowledge. Thus, ML researchers often rely on field experts to label such images; however, this process is costly and time consuming. Therefore, producing the large volume of labels required to develop flourishing deep networks turns out to be unfeasible.
	%CBMA might be a good addition to my algorithm seems to imrpove overall performance
\begin{abstract}
	
\end{abstract}
	\maketitle

\section{Introduction}
	A common issue of High-resolution peripheral quantitative computed tomography (HR-pQCT) scans is the appearance of motion artifacts in Images. These artifacts can appear due to involuntary movements like twitches and spasms. Depending on the severeness of those artifacts in the resulting image, it might not be sufficient for medical use and a re scan is necessary. The decision of the severity is currently done by a Doctor which gives the image a number from 1 to 5, where 1 equals no motion artifacts and 5 equals severe motion artifacts. The descission of severity is often biased and varies from doctor to doctor. To support the descission of the doctor there have been approaches by [] and []  to improve the confidence of the result. both of those methodes perform better than the average doctor but still have considerable error rates. In this paper we will propose a new Convolutional Neural Network(CNN) which uses state of the art methodes to calculate the severity of motion Scores in CT scans. Afterwards we will compare it to the two existing methodes.
	
\section{Literature review}
\subsection{Grad-CAM}

\section{Methodology}
\section{Results}
\section{Discussion/Conclusion}
\end{document}
