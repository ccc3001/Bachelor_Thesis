\documentclass[
a4paper, 
%a5paper,
%10pt,
%11pt,
12pt,
%twoside, % single sided printout
%oneside, % duplex printout (default)
%% binding correction is used to compensate for the paper lost during binding
%% of the document
%BCOR=0.7cm, % binding correction
%nobcorignoretitle, % do not ignore BCOR for title page
%% the following two options only concern the graphics included by the document
%% class
%grayscaletitle, % keep the title in grayscale
grayscalebody, % keep the rest of the document in grayscale
abstract=on,
%% expert options: your mileage may vary
%baseclass=scrreprt % special option to use a different document baseclass
twoside, BCOR10mm, 12pt, DIV13,headinclude, footexclude, final, abstracton, openright
]{ibireprt}


\usepackage[utf8]{inputenc}
%\usepackage[latin1]{inputenc}
\usepackage[T1]{fontenc}
%\usepackage{ngerman}
\usepackage[ngerman]{babel} %english,


%\usepackage{fancyhdr}
%%\pagestyle{fancy}
%%\lhead{\leftmark}
%\fancyhead[OR]{\thepage}% ungerade Seiten, rechts \thepage
%\fancyhead[OL]{\leftmark}% ungerade Seiten, links
%\fancyhead[ER]{\leftmark}% gerade Seiten, rechts
%\fancyhead[EL]{\thepage}% gerade Seiten, links \thepage
%\cfoot{}
%\pagestyle{fancy}

\renewcommand{\chaptermark}[1]{\markboth{#1}{}}
\renewcommand{\sectionmark}[1]{\markright{#1}{}}

\setlength{\parindent}{0em}
%\setlength{\parskip}{0mm}
\setcounter{tocdepth}{1}
\setcounter{secnumdepth}{2}
%\linespread{1.15}

\usepackage{quotchap}
\usepackage{nccmath}

\usepackage{microtype}

\usepackage{qtxmath,tgtermes}
\usepackage[scaled=.90]{helvet}
\usepackage{courier}

\usepackage{graphicx}
\usepackage{amsmath}
%\usepackage{amsfonts}
\usepackage{amssymb}
\usepackage{tabularx}
%\usepackage[bookmarks,plainpages=false]{hyperref} %colorlinks  ,urlbordercolor={111},linkbordercolor={111},citebordercolor={111}
\usepackage[Algorithmus]{algorithm}
\usepackage{algorithmic}
%\usepackage{tkmath}
\usepackage{exscale}
\usepackage{empheq}
\usepackage{color}
\usepackage{framed}
\usepackage{rotating}
\usepackage{longtable}
\usepackage[hang,small,bf]{caption}
\usepackage{booktabs}
\usepackage{colortbl}

\usepackage[babel,german=quotes]{csquotes}
\usepackage{ntheorem}
\usepackage{blindtext}

%\mathindent1.5cm
\def\fleq{\@fleqntrue \let\mathindent\@mathmargin \@mathmargin=\leftmargini}
\def\cneq{\@fleqnfalse}


%\setcapindent{0em}

\newenvironment{fshaded}{%
\def\FrameCommand{\fcolorbox{framecolor}{shadecolor}}%
\MakeFramed {\FrameRestore}}%
{\endMakeFramed}

\theoremseparator{:}

\newtheorem{theorem}{Theorem}[chapter]
\newtheorem{lemma}{Lemma}[chapter]
\newtheorem{remark}[theorem]{Bemerkung}
\newtheorem{definition}[theorem]{Definition}
\newtheorem{example}{Beispiel}
%\newtheorem{proof}[theorem]{Beweis}
\newtheorem{corollary}[theorem]{Corollary}

\newenvironment{Theorem}{\goodbreak \definecolor{shadecolor}{rgb}{0.95,0.95,0.95}%
\definecolor{framecolor}{rgb}{0,0,0}%
\begin{fshaded}\begin{theorem}\sl}{\end{theorem} \end{fshaded}}
\newenvironment{Lemma}{\goodbreak \definecolor{shadecolor}{rgb}{0.95,0.95,0.95}%
\definecolor{framecolor}{rgb}{0,0,0}%
\begin{fshaded} \begin{lemma}\sl}{\end{lemma} \end{fshaded}}
\newenvironment{Remark}{\goodbreak \begin{remark}\rm}{\hfill  $\square$\end{remark}}
\newenvironment{Example}{\goodbreak \begin{example}\rm}{\hfill $\square$ \end{example}}
%\newenvironment{Proof}{\goodbreak \begin{proof}\rm}{\hfill $\blacksquare$ \end{proof}}
\newenvironment{Definition}{\goodbreak \definecolor{shadecolor}{rgb}{0.95,0.95,0.95}%
\definecolor{framecolor}{rgb}{0,0,0}%
\begin{fshaded} \begin{definition}\rm}{\hfill  \end{definition} \end{fshaded} }
\newenvironment{Corollary}{\goodbreak \begin{corollary}\rm}{\end{corollary}}

\newenvironment{Proof}[1][Beweis:]{\begin{trivlist}
\item[\hskip \labelsep {\bfseries #1}]}{\hfill $\blacksquare$\end{trivlist}}

\numberwithin{equation}{chapter}
\numberwithin{table}{chapter}
\numberwithin{figure}{chapter}
\numberwithin{algorithm}{chapter}
\numberwithin{example}{chapter}
\numberwithin{example}{chapter}

\def\i{\mbox{\small{\rm i}}}
\def\ti{\mbox{\scriptsize{\rm i}}}
\newcommand{\e}[1]{{\rm e}^{ #1}}
\renewcommand{\mod}{\;{\rm mod}\;}

\newcommand{\zb}[1]{\mbox{\boldmath{${#1}$}}}
\newcommand{\zbs}[1]{\mbox{\boldmath\scriptsize{${#1}$}}}
\newcommand{\zbss}[1]{\mbox{\boldmath\tiny{${#1}$}}}

\newcommand{\adj}{{\ensuremath{\mathsf{H}}}}
\newcommand{\trans}{{\ensuremath{\mathsf{T}}}}


% Keine "Schusterjungen"
\clubpenalty = 10000
% Keine "Hurenkinder"
\widowpenalty = 10000 \displaywidowpenalty = 10000 %\displaywidowpenalty = 10000






% Information for the Titlepage
\author{Johann Strunck}
\title{Deep learning based grading of motionartifacts in HR-pQCT}
%\date{\today}
\date{\today}
\subject{Bachelor thesis}
\professor{Prof.~Dr.-Ing.~Tobias Knopp}
\advisor{Dr.~rer.~nat.~Martin Hofmann}







\begin{document}
%\frontmatter
\maketitle
%\mainmatter


\newpage
${}^{}$
\vfill
\noindent
Ich versichere an Eides statt, die vorliegende Arbeit selbstständig und nur unter Benutzung der angegebenen Quellen und Hilfsmittel angefertigt zu haben.\\
\vspace{1.5cm}

\noindent
Hamburg, den ??.??.2010
\thispagestyle{empty}
\newpage
\newpage

\setlength{\parskip}{1.5mm }

%\newpage

%\maketitle



\tableofcontents


\chapter*{Summary}
	A common issue of High-resolution peripheral quantitative computed tomography (HR-pQCT) scans is the appearance of motion artifacts in Images. These artifacts can appear due to involuntary movements like twitches and spasms. Depending on the severeness of those artifacts in the resulting image, it might not be sufficient for medical use and a re scan is necessary. The decision of the severity is decided by a Doctor which gives the image a number from 1 to 5, where 1 equals no motion artifacts and 5 equals severe motion artifacts. The descission of severity is often biased and varies from doctor to doctor. To support the descission of the doctor there have been approaches by [] and []  to improve the confidence of the result. both methodes can be performed with the absence of a doctor and results of [] show that with crossvalidation of another doctor a Convolutional Neural Network(CNN) can reach a higher accuracy than the cross validation of two doctors without a CNN. The CNN still has a considerable error rate. In this paper we will propose a new CNN which uses state of the art methods to calculate the severity of motion Scores in CT scans. Afterwards we will compare it to the two existing methods and show how those methods perform on the our data set

\chapter{Introduction}

High-resolution peripheral quantitative computed tomography (HR-pQCT) is a specialized non-invasive imaging technique that provides detailed and accurate three-dimensional images of bone and tissue microarchitecture at the peripheral skeletal sites.In our paper we work with images of the radius and thibia. This advanced imaging modality offers several distinct advantages. One advantage is that HR-pQCT provides high resolution images that allow a thorough assessment of a scanned bone micro architecture. It offers precise measurement of bone mineral density(BMD) and geometric parameters such as trabecular thickness and cortical thickness. HR-pQCT has applications in both clinical and research setting and can help make more informed decissions about patient management and treatment strategies. It can provide insight into fracture or the risk of its occurrence. HR-pQCT imaging requires the patient to remain still during the scan to avoid motion artifacts. This can be challenging for certain patient populations, such as children or individuals with limited  mobility. Depending on the severity of the motion artifact the scan must be repeated. To determine the severity of the scan [Bone(Maybe other paper)] Introduced a scale from 1 (no visible motion artefacts) to 5 (significant horizontal streaks). In Studies it is commonly implemented, that Scans with a grading of 4 or 5 have to be repeated to mitigate the effects of the motion artifacts. In our paper we will also use this scale of measurement.

%TODO: part about CNN's
In this paper we will introduce a Convolutional Neural Network Structure which is designed to predict the severity of the motion artifact and compare this Structure to the findings of [Sode] and []. 

A reoccurring issue in medical imaging is the lack of data for training in our case we had 500 labeled examples. If we compare this to the amount of data used in training state of the art networks it's a small fraction. This comes on the one hand from the fact that the labeling task in medical imaging can just be performed my professionals and therefore the labeling process is costly and just a few people can do it. Another issue is the availability of data since patient data cant be accessed and used as easy. Therefore we need to find a way to augment the data 


\chapter{Methods}

%Data Augmentation
A big issue in the space of medical imaging is the lack of training data. When training a CNN with to little data we either have to stop early and don't get a optimal accuracy for the network. If we would further train the network with the same samples the network would overfit and lose its validity  
therefore we nee to find a way to augment the data so that it still has the same meaning for a person. The concept of data augmentation is well spread in the medilcal imaging field 
to use data augmentaiton techniques like rotation 

\begin{align}
 \zb S &= \zb c \zb u
\end{align}

\chapter{Experimental Setup}



\chapter{Results}

\chapter{Discussion}

\bibliographystyle{unsrt} %unsrt abbrv
\bibliography{ref}

\end{document}
